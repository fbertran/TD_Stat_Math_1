% DST n1
% DST n1

\documentclass[11pt,a4paper]{article}
\usepackage{a4wide}\usepackage{amsmath,amssymb}
\usepackage{dsfont}
\usepackage[utf8]{inputenc} % entree 8 bits iso-latin1
\usepackage[T1]{fontenc}      % encodage 8 bits des fontes utilisees
\usepackage[french]{babel}%typo française
\usepackage{times}
\newcommand{\R}{\mathbb{R}}\newcommand{\C}{\mathbb{C}}
\newcommand{\N}{\mathbb{N}}\newcommand{\Q}{\mathbb{Q}}


\def \I{\mathbb{I}}
\def \N{\mathbb{N}}
\def \R{\mathbb{R}}
\def \M{\mathbb{M}}
\def \Z{\mathbb{Z}}
\def \E{\mathbb{E}}
\def \F{\mathbb{F}}
\def \P{\mathbb{P}}
\def \Q{\mathbb{Q}}
\def \D{\mathbb{D}}


\def \Ac{{\cal A}}
\def \Bc{{\cal B}}
\def \Cc{{\cal C}}
\def \Dc{{\cal D}}
\def \Ec{{\cal E}}
\def \Fc{{\cal F}}
\def \Gc{{\cal G}}
\def \Hc{{\cal H}}
\def \Ic{{\cal I}}
\def \Kc{{\cal K}}
\def \Lc{{\cal L}}
\def \Pc{{\cal P}}
\def \Qc{{\cal Q}}
\def \Mc{{\cal M}}
\def \Nc{{\cal N}}
\def \Oc{{\cal O}}
\def \Sc{{\cal S}}
\def \Tc{{\cal T}}
\def \Uc{{\cal U}}
\def \Vc{{\cal V}}
\def \Wc{{\cal W}}
\def \Yc{{\cal Y}}
\def \Zc{{\cal Z}}
\def \Xc{{\cal X}}






\def \PI{\displaystyle\Pi}

\def \Sum{\displaystyle\sum}
\def \Prod{\displaystyle\prod}
\def \Int{\displaystyle\int}
\def \Frac{\displaystyle\frac}
\def \Inf{\displaystyle\inf}
\def \Sup{\displaystyle\sup}
\def \Lim{\displaystyle\lim}
\def \Liminf{\displaystyle\liminf}
\def \Limsup{\displaystyle\limsup}
\def \Max{\displaystyle\max}
\def \Min{\displaystyle\min}




\def \ni{\noindent}

\def \eps{\varepsilon}


\def \ep{\hbox{ }\hfill$\Box$}


\def\Dt#1{\Frac{\partial #1}{\partial t}}
\def\Dx#1{\Frac{\partial #1}{\partial x}}
\def\Ds#1{\Frac{\partial #1}{\partial s}}
\def\Dss#1{\Frac{\partial^2 #1}{\partial s^2}}
\def\Dy#1{\Frac{\partial #1}{\partial y}}
\def\Dyy#1{\Frac{\partial^2 #1}{\partial y^2}}
\def\Dsy#1{\Frac{\partial^2 #1}{\partial s \partial y}}
\def\Dk#1{\Frac{\partial #1}{\partial k}}
\def\Dp#1{\Frac{\partial #1}{\partial p}}
\def\Dkk#1{\Frac{\partial^2 #1}{\partial k^2}}
\def\Dpp#1{\Frac{\partial^2 #1}{\partial p^2}}
\def\Dky#1{\Frac{\partial^2 #1}{\partial k \partial y}}
\def\Dkp#1{\Frac{\partial^2 #1}{\partial k \partial p}}
\def\Dyp#1{\Frac{\partial^2 #1}{\partial y \partial p}}

\def\Dth#1{\Frac{\partial #1}{\partial \theta}}
\def\Dthi#1{\Frac{\partial #1}{\partial \theta_i}}
\def\Dthj#1{\Frac{\partial #1}{\partial \theta_j}}
\def\Dtth#1{\Frac{\partial^2 #1}{\partial \theta^2}}
\def\Dthij#1{\Frac{\partial^2 #1}{\partial \theta_i \partial \theta_j}}

\def\Dth#1{\Frac{\partial #1}{\partial \theta}}
\def\Dtth#1{\Frac{\partial^2 #1}{\partial \theta^2}}

\def\Dlam#1{\Frac{\partial #1}{\partial \lambda}}

\def\reff#1{{\rm(\ref{#1})}}

\def\beqs{\begin{eqnarray*}}
\def\enqs{\end{eqnarray*}}
\def\beq{\begin{eqnarray}}
\def\enq{\end{eqnarray}}






%%%%%\setbeamercovered{dynamic}






\newcounter{exo}
\def\cit{\addtocounter{exo}{-1}\refstepcounter{exo}\label}
\def\exo{\mbox{}\\[0em]\hspace*{0em}\bf Exercice
\addtocounter{exo}{1}\arabic{exo}.\rm\hspace{1ex}}




\begin{document}
\centerline{\sc \MSA}  \centerline{~}
\vskip1cm \centerline{{\bf TD 4}} \centerline{2018}

\exo Soit $X$ une variable al\'eatoire \`a valeurs dans $\N^*$
d\'efinie comme l'instant de premier succ\`es dans un sch\'ema de
Bernoulli de param\`etre $q \in ]0, 1[$.

\vspace{3mm}

1. V\'erifier que la loi de $X$ est une loi g\'eom\'etrique dont
on pr\'ecisera le param\`etre.

\vspace{3mm}

2. V\'erifier qu'il s'agit d'un mod\`ele exponentiel. Donner une
statistique exhaustive.

\vspace{3mm}

3. D\'eterminer I(q), l'information de Fisher sur $q$ d'un
\'echantillon de taille 1.

\vspace{3mm}

Soit $X_1,...,X_n$ un \'echantillon ind\'ependant de taille $n$ de
m\^eme loi que $X$.

\vspace{3mm}

4. D\'eterminer $\hat q_n$, l'estimateur du maximum de
vraisemblance de $q$.

\vspace{3mm}

5. Montrer que l'estimateur du maximum de vraisemblance est
asymptotiquement normal.

\vspace{3mm}

\exo On consid\`ere le mod\`ele d'\'echantillonnage $X_1 ...,X_n$
de taille $n$ associ\'e \`a la famille de lois exponentielles $P =
\{\Ec(\lambda), \lambda>0\}$. On veut estimer $\lambda$

\vspace{3mm}

1. A partir de la m\'ethode des moments, construire un estimateur
convergent $\hat \lambda_n$ de $\lambda$.

\vspace{3mm}

2. V\'erifier qu'il s'agit de l'estimateur du maximum de
vraisemblance.

\vspace{3mm}

3. D\'eterminer la loi de $\Sigma_{i=1}^n  X_i$ . Calculer
$E_{\lambda}[\hat \lambda_n]$. L'estimateur est-il sans biais?

\vspace{3mm}

4. D\'eterminer un estimateur $\hat \lambda_n^*$ sans biais et un
estimateur $\hat \lambda_n^o$ qui minimise le risque quadratique
parmi les estimateurs $$ \hat \lambda_n^{(c)} =
\frac{c}{\Sigma_{i=1}^n  X_i}, \: \hbox{où} \: c> 0$$


5. Calculer  l'information de Fisher et la borne Cramer-Rao pour un ESB de $\lambda$.

6. Les estimateurs \'etudi\'es font intervenir la statistique $S_n
= \sum_{i=1}^n X_i$. Est-elle exhaustive?


7. R\'esum\'e : quelles propri\'et\'es $\hat \lambda_n^*$ a-t-il
parmi les suivantes ?

(a) Sans biais. (b) Efficace. (c) Asymptotiquement normal.





\end{document}

% DST n1
% DST n1

\documentclass[11pt,a4paper]{article}
\usepackage{a4wide}\usepackage{amsmath,amssymb}
\usepackage{dsfont}
\usepackage[utf8]{inputenc} % entree 8 bits iso-latin1
\usepackage[T1]{fontenc}      % encodage 8 bits des fontes utilisees
\usepackage[french]{babel}%typo française
\usepackage{times}
\newcommand{\R}{\mathbb{R}}\newcommand{\C}{\mathbb{C}}
\newcommand{\N}{\mathbb{N}}\newcommand{\Q}{\mathbb{Q}}


\def \I{\mathbb{I}}
\def \N{\mathbb{N}}
\def \R{\mathbb{R}}
\def \M{\mathbb{M}}
\def \Z{\mathbb{Z}}
\def \E{\mathbb{E}}
\def \F{\mathbb{F}}
\def \P{\mathbb{P}}
\def \Q{\mathbb{Q}}
\def \D{\mathbb{D}}


\def \Ac{{\cal A}}
\def \Bc{{\cal B}}
\def \Cc{{\cal C}}
\def \Dc{{\cal D}}
\def \Ec{{\cal E}}
\def \Fc{{\cal F}}
\def \Gc{{\cal G}}
\def \Hc{{\cal H}}
\def \Ic{{\cal I}}
\def \Kc{{\cal K}}
\def \Lc{{\cal L}}
\def \Pc{{\cal P}}
\def \Qc{{\cal Q}}
\def \Mc{{\cal M}}
\def \Nc{{\cal N}}
\def \Oc{{\cal O}}
\def \Sc{{\cal S}}
\def \Tc{{\cal T}}
\def \Uc{{\cal U}}
\def \Vc{{\cal V}}
\def \Wc{{\cal W}}
\def \Yc{{\cal Y}}
\def \Zc{{\cal Z}}
\def \Xc{{\cal X}}






\def \PI{\displaystyle\Pi}

\def \Sum{\displaystyle\sum}
\def \Prod{\displaystyle\prod}
\def \Int{\displaystyle\int}
\def \Frac{\displaystyle\frac}
\def \Inf{\displaystyle\inf}
\def \Sup{\displaystyle\sup}
\def \Lim{\displaystyle\lim}
\def \Liminf{\displaystyle\liminf}
\def \Limsup{\displaystyle\limsup}
\def \Max{\displaystyle\max}
\def \Min{\displaystyle\min}




\def \ni{\noindent}

\def \eps{\varepsilon}


\def \ep{\hbox{ }\hfill$\Box$}


\def\Dt#1{\Frac{\partial #1}{\partial t}}
\def\Dx#1{\Frac{\partial #1}{\partial x}}
\def\Ds#1{\Frac{\partial #1}{\partial s}}
\def\Dss#1{\Frac{\partial^2 #1}{\partial s^2}}
\def\Dy#1{\Frac{\partial #1}{\partial y}}
\def\Dyy#1{\Frac{\partial^2 #1}{\partial y^2}}
\def\Dsy#1{\Frac{\partial^2 #1}{\partial s \partial y}}
\def\Dk#1{\Frac{\partial #1}{\partial k}}
\def\Dp#1{\Frac{\partial #1}{\partial p}}
\def\Dkk#1{\Frac{\partial^2 #1}{\partial k^2}}
\def\Dpp#1{\Frac{\partial^2 #1}{\partial p^2}}
\def\Dky#1{\Frac{\partial^2 #1}{\partial k \partial y}}
\def\Dkp#1{\Frac{\partial^2 #1}{\partial k \partial p}}
\def\Dyp#1{\Frac{\partial^2 #1}{\partial y \partial p}}

\def\Dth#1{\Frac{\partial #1}{\partial \theta}}
\def\Dthi#1{\Frac{\partial #1}{\partial \theta_i}}
\def\Dthj#1{\Frac{\partial #1}{\partial \theta_j}}
\def\Dtth#1{\Frac{\partial^2 #1}{\partial \theta^2}}
\def\Dthij#1{\Frac{\partial^2 #1}{\partial \theta_i \partial \theta_j}}

\def\Dth#1{\Frac{\partial #1}{\partial \theta}}
\def\Dtth#1{\Frac{\partial^2 #1}{\partial \theta^2}}

\def\Dlam#1{\Frac{\partial #1}{\partial \lambda}}

\def\reff#1{{\rm(\ref{#1})}}

\def\beqs{\begin{eqnarray*}}
\def\enqs{\end{eqnarray*}}
\def\beq{\begin{eqnarray}}
\def\enq{\end{eqnarray}}






%%%%%\setbeamercovered{dynamic}






\newcounter{exo}
\def\cit{\addtocounter{exo}{-1}\refstepcounter{exo}\label}
\def\exo{\mbox{}\\[0em]\hspace*{0em}\bf Exercice
\addtocounter{exo}{1}\arabic{exo}.\rm\hspace{1ex}}

%\parindent 0pt

%

\begin{document}
\centerline{\sc MSA}  \centerline{~}
\vskip1cm \centerline{{\bf TD 5}} \centerline{2018}

\exo Soit $X$ une v.a. de densité : $$f(x) = \frac{1}{2}
e^{-|x-\theta|}, x \in  \R$$ où $\theta$ est un paramètre réel
inconnu.

\vspace{2mm}

1. Calculer $E_{\theta}[X]$ et $\hbox{Var}_{\theta}[X]$. En
déduire un estimateur $T_n$ de $\theta$.

\vspace{2mm}

2. Construire un intervalle de confiance de niveau asymptotique
$95\%$ pour $\theta$ dans le cas où $n = 200$.


\exo On rappelle que dans le modèle uniforme $\Pc = \{\Uc [0,
\theta], \theta >0 \}$, l'Estimateur du Maximum de Vraisemblance
de $\theta$ est $\hat \theta_n = \max(X_1,...,X_n).$

\vspace{2mm}

{\bf 1.} Pour $x \in \R$, calculer $\P_{\theta}(\frac{\hat
\theta_n}{\theta} \leq x)$ et en déduire que la loi de $\frac{\hat
\theta_n}{\theta}$ ne dépend pas de $\theta$.

\vspace{2mm}

{\bf 2.} Construire un intervalle de confiance de niveau $1 -
\alpha$ pour $\theta$.

\exo Soit $X_1,...,X_n$ un échantillon de loi
$$ f(x, \theta) \;=\; e^{-(x- \theta)} {\bf 1}_{[\theta, \infty[}(x)$$

\vspace{2mm}

1. Donner la vraisemblance associée à l'échantillon ci-dessus.

\vspace{2mm}

2. Déterminer l'estimateur du maximum de vraisemblance $\hat
\theta_n$.

\vspace{2mm}

3. Déterminer la loi de $\hat \theta_n$. Est-il un estimateur sans
biais? asymptotiquement sans biais?

\vspace{2mm}

4. Montrer que $T_n = \hat \theta_n - \theta$ est une statistique
libre pour $\theta$. Déterminer sa loi.

\vspace{2mm}

5. Construire un intervalle de confiance de $\theta$ à un niveau
de confiance $\alpha \in [0,1]$.

\vspace{2mm}

6. Montrer que $\hat T_n = n \left(1 - e^{-(\hat \theta_n -
\theta)} \right)$ est une statistique libre pour $\theta$.
Déterminer sa loi.

\vspace{2mm}

7. En déduire un intervalle de confiance asymptotique pour
$\theta$ avec un niveau de confiance $\alpha$.


\exo Les lecteurs de tension art\'erielle systolique (TAS) (en mm Hg) sur un individu \a la m\^eme heure pendant 7 jours cons\'ecutifs ont fournis les donn\'ees suivantes
\[
 \begin{array}{c*{8}c}
{\rm jour}  &  1   & 2  & 3   &  4    & 5  & 6  & 7  \\
 x_i &  161 & 155   & 142  &157  & 150  &  192 & 156 
 \end{array}  
\]
\begin{enumerate}
\item Calculer la moyenne empirique et la m\'ediane de l'\'echantillon.
\item En faisant l'hypoth\`ese que la mesure de tension $X$ suit une loi $\mathcal{N}(\mu,\sigma^2)$, avec $\sigma^2=100$, donner un intervalle de confiance (IC) (bilat\'eral) \`a $95 \%$ pour $\mu$.
\item Combien de jours faut-il observer la TSA pour que la longueur de l'IC \`a $95\%$ n'exc\`ede pas 5mm de Hg.
\item Que devient l'IC calcul\'e pour r\'epondre \`a  la question  $(b)$ si on suppose que $\sigma^2$ est inconnue.
\item Donner un intervalle de confiance unilat\'eral (de la forme $[0,U]$) \`a $95\%$ sur $\sigma^2$ en supposant $\mu=160$, puis en supposant $\mu$ inconnu.
\end{enumerate}

\end{document}

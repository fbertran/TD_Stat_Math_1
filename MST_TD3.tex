% DST n1

\documentclass[11pt,a4paper]{article}
\usepackage{a4wide}\usepackage{amsmath,amssymb}
\usepackage{dsfont}
\usepackage[utf8]{inputenc} % entree 8 bits iso-latin1
\usepackage[T1]{fontenc}      % encodage 8 bits des fontes utilisees
\usepackage[french]{babel}%typo française
\usepackage{times}
\newcommand{\R}{\mathbb{R}}\newcommand{\C}{\mathbb{C}}
\newcommand{\N}{\mathbb{N}}\newcommand{\Q}{\mathbb{Q}}


\def \I{\mathbb{I}}
\def \N{\mathbb{N}}
\def \R{\mathbb{R}}
\def \M{\mathbb{M}}
\def \Z{\mathbb{Z}}
\def \E{\mathbb{E}}
\def \F{\mathbb{F}}
\def \P{\mathbb{P}}
\def \Q{\mathbb{Q}}
\def \D{\mathbb{D}}


\def \Ac{{\cal A}}
\def \Bc{{\cal B}}
\def \Cc{{\cal C}}
\def \Dc{{\cal D}}
\def \Ec{{\cal E}}
\def \Fc{{\cal F}}
\def \Gc{{\cal G}}
\def \Hc{{\cal H}}
\def \Ic{{\cal I}}
\def \Kc{{\cal K}}
\def \Lc{{\cal L}}
\def \Pc{{\cal P}}
\def \Qc{{\cal Q}}
\def \Mc{{\cal M}}
\def \Nc{{\cal N}}
\def \Oc{{\cal O}}
\def \Sc{{\cal S}}
\def \Tc{{\cal T}}
\def \Uc{{\cal U}}
\def \Vc{{\cal V}}
\def \Wc{{\cal W}}
\def \Yc{{\cal Y}}
\def \Zc{{\cal Z}}
\def \Xc{{\cal X}}






\def \PI{\displaystyle\Pi}

\def \Sum{\displaystyle\sum}
\def \Prod{\displaystyle\prod}
\def \Int{\displaystyle\int}
\def \Frac{\displaystyle\frac}
\def \Inf{\displaystyle\inf}
\def \Sup{\displaystyle\sup}
\def \Lim{\displaystyle\lim}
\def \Liminf{\displaystyle\liminf}
\def \Limsup{\displaystyle\limsup}
\def \Max{\displaystyle\max}
\def \Min{\displaystyle\min}




\def \ni{\noindent}

\def \eps{\varepsilon}


\def \ep{\hbox{ }\hfill$\Box$}


\def\Dt#1{\Frac{\partial #1}{\partial t}}
\def\Dx#1{\Frac{\partial #1}{\partial x}}
\def\Ds#1{\Frac{\partial #1}{\partial s}}
\def\Dss#1{\Frac{\partial^2 #1}{\partial s^2}}
\def\Dy#1{\Frac{\partial #1}{\partial y}}
\def\Dyy#1{\Frac{\partial^2 #1}{\partial y^2}}
\def\Dsy#1{\Frac{\partial^2 #1}{\partial s \partial y}}
\def\Dk#1{\Frac{\partial #1}{\partial k}}
\def\Dp#1{\Frac{\partial #1}{\partial p}}
\def\Dkk#1{\Frac{\partial^2 #1}{\partial k^2}}
\def\Dpp#1{\Frac{\partial^2 #1}{\partial p^2}}
\def\Dky#1{\Frac{\partial^2 #1}{\partial k \partial y}}
\def\Dkp#1{\Frac{\partial^2 #1}{\partial k \partial p}}
\def\Dyp#1{\Frac{\partial^2 #1}{\partial y \partial p}}

\def\Dth#1{\Frac{\partial #1}{\partial \theta}}
\def\Dthi#1{\Frac{\partial #1}{\partial \theta_i}}
\def\Dthj#1{\Frac{\partial #1}{\partial \theta_j}}
\def\Dtth#1{\Frac{\partial^2 #1}{\partial \theta^2}}
\def\Dthij#1{\Frac{\partial^2 #1}{\partial \theta_i \partial \theta_j}}

\def\Dth#1{\Frac{\partial #1}{\partial \theta}}
\def\Dtth#1{\Frac{\partial^2 #1}{\partial \theta^2}}

\def\Dlam#1{\Frac{\partial #1}{\partial \lambda}}

\def\reff#1{{\rm(\ref{#1})}}

\def\beqs{\begin{eqnarray*}}
\def\enqs{\end{eqnarray*}}
\def\beq{\begin{eqnarray}}
\def\enq{\end{eqnarray}}






%%%%%\setbeamercovered{dynamic}






\newcounter{exo}
\def\cit{\addtocounter{exo}{-1}\refstepcounter{exo}\label}
\def\exo{\mbox{}\\[0em]\hspace*{0em}\bf Exercice
\addtocounter{exo}{1}\arabic{exo}.\rm\hspace{1ex}}



\begin{document}
\centerline{\sc MSA}  \centerline{~}
\vskip1cm \centerline{{\bf TD 3}} \centerline{2018}


\exo On observe la r\'ealisation d'un \'echantillon $X_1,...,X_n$
de taille $n$ de loi $P_{\theta}$ de densit\'e $$ f(x, \theta) =
\frac{1}{\theta} (1 -x )^{\frac{1}{\theta} - 1} {\bf 1}_{]0, 1[}
(x), \:\theta \in \R_+^*.$$

1. Donner une statistique exhaustive.

2. D\'eterminer l'estimateur du maximum de vraisemblance $T_n$ de
$\theta$.

3. Montrer que $-\ln(1 -X_i)$ suit une loi exponentielle dont on
pr\'ecisera le param\`etre.

4. Calculer le biais et le risque quadratique de $T_n$. Cet
estimateur est-il convergent?



\exo 1. Soit $(X_1,...,X_n)$ un \'echantillon de $n$ variables
i.i.d. de loi de Poisson de param\`etre $\lambda$:

$$P(X_1 = k) = e^{-\lambda} \frac{\lambda^k}{k!},\: ; k \in \N^*$$
Calculer la vraisemblance de l'\'echantillon, d\'eterminer si le
mod\`ele est exponentiel et exhiber une statistique exhaustive.

2. M\^emes questions avec une loi de Pareto de param\`etres
$\alpha$ et $\theta$ avec $\alpha > 1$, $\theta > 0$ de densit\'e

$$f(x) = \frac{\alpha -1}{\theta} (\frac{\theta}{x})^{\alpha} {\bf 1}_{[\theta,
\infty[}(x).$$

3. M\^emes questions avec une loi de Weibull de param\`etres
$\alpha$ et $\theta$ avec $\alpha > 1$, $\theta > 0$ de densit\'e

$$f(x) = \alpha \theta x^{\alpha -1} e^{- \theta x^{\alpha}} 1_{[0,
\infty[}(x).$$

On distinguera le cas $\alpha$ inconnu du cas $\alpha$ connu.

4. M\^emes questions avec une loi uniforme sur $[0, \theta]$ avec
$\theta > 0$ inconnu.


\exo Calculer l'information de Fisher dans les mod\`eles
statistiques suivants :

1. Une loi de Poisson de param\`etre $\lambda$:

$$P(X_1 = k) = e^{-\lambda} \frac{\lambda^k}{k!},\: ; k \in \N^*$$

2. Une loi de Pareto de param\`etres $\alpha$ et $\theta$ avec
$\alpha > 1$, $\theta > 0$ de densit\'e

$$f(x) = \frac{\alpha -1}{\theta} (\frac{\theta}{x})^{\alpha} {\bf 1}_{[\theta,
\infty[}(x).$$

3. Une loi de Weibull de param\`etres $\alpha$ et $\theta$ avec
$\alpha
> 1$, $\theta > 0$ de densit\'e

$$f(x) = \alpha \theta x^{\alpha -1} e^{- \theta x^{\alpha}} 1_{[0,
\infty[}(x).$$

4. Une loi uniforme sur $[0; \theta]$ avec $\theta > 0$ inconnu.




\end{document}


\documentclass[11pt,a4paper]{article}
\usepackage{a4wide}\usepackage{amsmath,amssymb}
\usepackage{dsfont}
\usepackage[utf8]{inputenc} % entree 8 bits iso-latin1
\usepackage[T1]{fontenc}      % encodage 8 bits des fontes utilisees
\usepackage[french]{babel}%typo française
\usepackage{times}
\newcommand{\R}{\mathbb{R}}\newcommand{\C}{\mathbb{C}}
\newcommand{\N}{\mathbb{N}}\newcommand{\Q}{\mathbb{Q}}
\usepackage{color}

\def \I{\mathbb{I}}
\def \N{\mathbb{N}}
\def \R{\mathbb{R}}
\def \M{\mathbb{M}}
\def \Z{\mathbb{Z}}
\def \E{\mathbb{E}}
\def \F{\mathbb{F}}
\def \P{\mathbb{P}}
\def \Q{\mathbb{Q}}
\def \D{\mathbb{D}}


\def \Ac{{\cal A}}
\def \Bc{{\cal B}}
\def \Cc{{\cal C}}
\def \Dc{{\cal D}}
\def \Ec{{\cal E}}
\def \Fc{{\cal F}}
\def \Gc{{\cal G}}
\def \Hc{{\cal H}}
\def \Ic{{\cal I}}
\def \Kc{{\cal K}}
\def \Lc{{\cal L}}
\def \Pc{{\cal P}}
\def \Rc{{\cal R}}
\def \Qc{{\cal Q}}
\def \Mc{{\cal M}}
\def \Nc{{\cal N}}
\def \Oc{{\cal O}}
\def \Sc{{\cal S}}
\def \Tc{{\cal T}}
\def \Uc{{\cal U}}
\def \Vc{{\cal V}}
\def \Wc{{\cal W}}
\def \Yc{{\cal Y}}
\def \Zc{{\cal Z}}
\def \Xc{{\cal X}}






\def \PI{\displaystyle\Pi}

\def \Sum{\displaystyle\sum}
\def \Prod{\displaystyle\prod}
\def \Int{\displaystyle\int}
\def \Frac{\displaystyle\frac}
\def \Inf{\displaystyle\inf}
\def \Sup{\displaystyle\sup}
\def \Lim{\displaystyle\lim}
\def \Liminf{\displaystyle\liminf}
\def \Limsup{\displaystyle\limsup}
\def \Max{\displaystyle\max}
\def \Min{\displaystyle\min}




\def \ni{\noindent}

\def \eps{\varepsilon}


\def \ep{\hbox{ }\hfill$\Box$}


\def\Dt#1{\Frac{\partial #1}{\partial t}}
\def\Dx#1{\Frac{\partial #1}{\partial x}}
\def\Ds#1{\Frac{\partial #1}{\partial s}}
\def\Dss#1{\Frac{\partial^2 #1}{\partial s^2}}
\def\Dy#1{\Frac{\partial #1}{\partial y}}
\def\Dyy#1{\Frac{\partial^2 #1}{\partial y^2}}
\def\Dsy#1{\Frac{\partial^2 #1}{\partial s \partial y}}
\def\Dk#1{\Frac{\partial #1}{\partial k}}
\def\Dp#1{\Frac{\partial #1}{\partial p}}
\def\Dkk#1{\Frac{\partial^2 #1}{\partial k^2}}
\def\Dpp#1{\Frac{\partial^2 #1}{\partial p^2}}
\def\Dky#1{\Frac{\partial^2 #1}{\partial k \partial y}}
\def\Dkp#1{\Frac{\partial^2 #1}{\partial k \partial p}}
\def\Dyp#1{\Frac{\partial^2 #1}{\partial y \partial p}}

\def\Dth#1{\Frac{\partial #1}{\partial \theta}}
\def\Dthi#1{\Frac{\partial #1}{\partial \theta_i}}
\def\Dthj#1{\Frac{\partial #1}{\partial \theta_j}}
\def\Dtth#1{\Frac{\partial^2 #1}{\partial \theta^2}}
\def\Dthij#1{\Frac{\partial^2 #1}{\partial \theta_i \partial \theta_j}}

\def\Dth#1{\Frac{\partial #1}{\partial \theta}}
\def\Dtth#1{\Frac{\partial^2 #1}{\partial \theta^2}}

\def\Dlam#1{\Frac{\partial #1}{\partial \lambda}}

\def\reff#1{{\rm(\ref{#1})}}

\def\beqs{\begin{eqnarray*}}
\def\enqs{\end{eqnarray*}}
\def\beq{\begin{eqnarray}}
\def\enq{\end{eqnarray}}


\usepackage[french]{babel}%typo française

\usepackage{times}




%%%%%\setbeamercovered{dynamic}






\newcounter{exo}
\def\cit{\addtocounter{exo}{-1}\refstepcounter{exo}\label}
\def\exo{\mbox{}\\[0em]\hspace*{0em}\bf Exercice
\addtocounter{exo}{1}\arabic{exo}.\rm\hspace{1ex}}

\parindent 0pt

%

\begin{document}
\centerline{\sc MSA}  \centerline{~}
\vskip1cm \centerline{{\bf TD 6}} \centerline{2018}

\exo
On a relevé pendant 30 jours ouvrables consécutifs les nombres quotidiens d'actes de délinquance commis dans un  centre commercial. Les résultats sont reportés dans le tableau suivant où $n_k$ désigne le nombre de jours et  $k$ le nombre d'actes de délinquance par milliers de personnes qui ont été commis :

\begin{center}
  \begin{tabular}{c | c c c c c c}
    $k$ & 0 & 1& 2& 3& 4& 5\\
    \hline
    $n_k$ & 4&10&11&1&1&3
  \end{tabular}
\end{center}

On suppose que le nombre $X_i$ d'actes de délinquance par milliers de personnes commis le jour $i$, $1 \leqslant i \leqslant 30$, dans  le  centre commercial  suit  une  loi  de Poisson  de  paramètre $\lambda$   inconnu et que les $X_1,\ldots,X_{30}$ sont indépendantes.  D'après   les  statistiques   du   ministère  de
l'intérieur, il  se produit  en moyenne 1,5  actes de  délinquance par milliers de personnes par
jour dans chaque centre commercial de la région parisienne. Ce chiffre
est   considéré    comme   sous-estimé   par    une   association   de
commer\c{c}ants. Nous allons  utiliser la théorie des tests  pour tenter de
trancher cette question.
\vskip 0.2cm
{\bf 1.} On considère le problème de test suivant:
 $$  \left \{ \begin{array}{ll}
    H_0: & \quad \lambda = \lambda_0\\
    H_1: & \quad \lambda = \lambda_1, 
     \end{array}  
 \right.
 $$
  avec $\lambda_1>\lambda_0$. En utilisant le théorème de
  Neymann-Pearson, montrer que le test optimal au niveau $\alpha$
  pour ce problème s'exprime en fonction de la statistique
  $\bar{X}$. Donner la forme de la région critique du test.
\vskip 0.2cm
{\bf 2.} En supposant $n$ grand, déterminer l'expression littérale de la
  région critique.  Décide-t-on du rejet ou non de l'hypothèse $H_0$   lorsque  $\lambda_0=1.5$
  et $\alpha =0.05$?
\vskip 0.2cm
{\bf 3.} Calculer la puissance du test pour $\lambda_1=2$.
\vskip 0.2cm
{\bf 4.}  On considère cette fois les hypothèses suivantes:
$$  \left \{ \begin{array}{ll}
    H_0: & \quad \lambda = \lambda_0\\
    H_1: & \quad \lambda > \lambda_0 \;.
    \end{array}  
 \right.
 $$
Quelle est la région critique de ce test? Y a-t-il lieu de remettre en
cause les statistiques du ministère de l'intérieur?

\vskip 0.3cm


\exo Soit $(X_n, n \geq 1)$, une suite de variables aléatoires
indépendantes, de loi normale $N(\theta, \theta)$, avec $\theta>
0$. L'objectif de cet exercice est de présenter deux tests pour
déterminer si pour une valeur déterminée $\theta_0 > 0$, on a
$\theta = \theta_0$ (hypothèse $H_0$) ou $\theta > \theta_0$
(hypothèse $H_1$). On note
$$ \bar X_n = \frac{1}{n} \Sum_{i=1}^n X_i,\: \; V_{n} =
\frac{1}{n-1}\Sum_{i=1}^n (X_i - \bar X_n)^2 \;=\;
\frac{1}{n-1}\Sum_{i=1}^n X_i^2 - \frac{n}{n-1} \bar X_n^2
$$
\ni {\bf 1.} Déterminer $\hat \theta_n$, l'estimateur du maximum
de vraisemblance de $\theta$. Montrer directement qu'il est
convergent, asymptotiquement normal et donner sa variance
asymptotique. Est-il asymptotiquement efficace ?

\vspace{1mm}

\ni {\bf 2.} Construire un test asymptotique convergent à l'aide
de l'estimateur du maximum de vraisemblance.

\vspace{1mm}

\ni On considère la classe des estimateurs $T_n^{\lambda}$ de la
forme $T_n^{\lambda} = \lambda \bar X_n + (1- \lambda) V_n$.

\vspace{1mm}

 \ni {\bf 3.} Montrer que la suite d'estimateurs
$(T^{\lambda}_n, n \geq 2)$ est convergente, sans biais. Donner la
variance de $T^{\lambda}_n$ .



\exo L'information dans une direction de l'espace prise par un
radar de surveillance aérienne se présente sous la forme d'un
n-échantillon $X = (X_1,...,X_n)$ de variables aléatoires
indépendantes de même loi gaussienne de moyenne $\theta$,
paramètre inconnu, et de variance $\sigma^2$ connu. On notera
$f(x,\theta)$, $x \in \R^n$ la densité du vecteur aléatoire.

\vspace{1mm}

\ni En l'absence de tout Objet Volant (Hypothèse $H_0$) $\theta =
\theta_0 \in \R^+$, sinon (Hypothèse $H_1$),$\theta = \theta_1$,
avec $\theta_1 > \theta_0$.

\vspace{1mm}

{\bf 1.} Montrer comment le lemme de Neyman-Pearson permet la
construction d'un test de l'hypothèse $\theta = \theta_0$ contre
l'hypothèse $\theta = \theta_1$ de niveau $\alpha$ et de puissance
maximale.

\vspace{1mm}

{\bf 2.} Quelle est la plus petite valeur de $n$ permettant de
construire un test de niveau $\alpha$, $\alpha \in  [0, 1]$ et
d'erreur de deuxième espèce inférieure ou égale à $\beta$, $\beta
\in [0, 1]$, avec $\alpha < \beta$?

\vspace{1mm}

{\bf 3.} Supposons maintenant qu'en presence d'objet volant
l'information fournie par le radar est un n-échantillon $X =
(X_1,...,X_n)$ de variables aléatoires indépendantes de même loi
gaussienne de moyenne $\theta \neq \theta_0$, $\theta \in \R$ et
de variance $\sigma^2$. Peut-on construire un test de l'hypothèse
$\theta = \theta_0$ contre l'hypothèse $\theta \neq \theta_0$ de
niveau $\alpha$ donné uniformément plus puissant?

\vspace{1mm}

\exo Une agence de voyage souhaite cibler sa clientèle. Elle sait
que les coordonnées du lieu de vie d'un client $(X,Y)$ rapportées
au lieu de naissance $(0,0)$ sont une information significative
pour connaître le goût de ce client. Elle distingue :


\vspace{1mm}

\ni - La population 1 (Hypothèse $H_0$) dont la loi de répartition
a pour densité : \beqs p_1(x,y) = \frac{1}{2 \pi}
e^{-\frac{x^2+y^2}{2}} \enqs

\vspace{1mm}

\ni - La population 2 (Hypothèse $H_1$) dont la loi de répartition
a pour densité : \beqs p_2(x,y) = \frac{1}{16} {\bf 1}_{[-2,2]}(x)
{\bf 1}_{[-2,2]}(y) \enqs

\vspace{1mm}

\ni L'agence souhaite tester l'hypothèse qu'un nouveau client
vivant en $(x, y)$ appartient à la population 1 plutôt qu'à la
population 2.

\vspace{1mm}

\ni {\bf 1.} Proposer un test de niveau inférieur à $\alpha = 5\%$
et de puissance maximale, construit à partir du rapport de
vraisemblance.

\vspace{1mm}


\ni {\bf 2.} Donner une statistique de test et caractériser
graphiquement la région critique dans $\R^2$.


\exo Soit $X_1,...,X_n$ un n-échantillon de loi exponentielle de
paramètre $\frac{1}{\theta} > 0$.

\vspace{1mm}

\ni {\bf  1.} Construire le test de niveau $\alpha$ $H_0 =
\{\theta = \theta_0\}$ contre $H_1 = \{\theta > \theta_0\}$.

\vspace{1mm}

\ni {\bf 2.} Construire le test de niveau $\alpha$ $H_0 = \{\theta
= \theta_0\}$ contre $H_1 = \{\theta \neq \theta_0\}$.


\end{document}


\documentclass[11pt,a4paper]{article}
\usepackage{a4wide}\usepackage{amsmath,amssymb}
\usepackage{dsfont}
\usepackage[utf8]{inputenc} % entree 8 bits iso-latin1
\usepackage[T1]{fontenc}      % encodage 8 bits des fontes utilisees
\usepackage[french]{babel}%typo française
\usepackage{times}
\newcommand{\R}{\mathbb{R}}\newcommand{\C}{\mathbb{C}}
\newcommand{\N}{\mathbb{N}}\newcommand{\Q}{\mathbb{Q}}
\usepackage{color}

\def \I{\mathbb{I}}
\def \N{\mathbb{N}}
\def \R{\mathbb{R}}
\def \M{\mathbb{M}}
\def \Z{\mathbb{Z}}
\def \E{\mathbb{E}}
\def \F{\mathbb{F}}
\def \P{\mathbb{P}}
\def \Q{\mathbb{Q}}
\def \D{\mathbb{D}}


\def \Ac{{\cal A}}
\def \Bc{{\cal B}}
\def \Cc{{\cal C}}
\def \Dc{{\cal D}}
\def \Ec{{\cal E}}
\def \Fc{{\cal F}}
\def \Gc{{\cal G}}
\def \Hc{{\cal H}}
\def \Ic{{\cal I}}
\def \Kc{{\cal K}}
\def \Lc{{\cal L}}
\def \Pc{{\cal P}}
\def \Rc{{\cal R}}
\def \Qc{{\cal Q}}
\def \Mc{{\cal M}}
\def \Nc{{\cal N}}
\def \Oc{{\cal O}}
\def \Sc{{\cal S}}
\def \Tc{{\cal T}}
\def \Uc{{\cal U}}
\def \Vc{{\cal V}}
\def \Wc{{\cal W}}
\def \Yc{{\cal Y}}
\def \Zc{{\cal Z}}
\def \Xc{{\cal X}}






\def \PI{\displaystyle\Pi}

\def \Sum{\displaystyle\sum}
\def \Prod{\displaystyle\prod}
\def \Int{\displaystyle\int}
\def \Frac{\displaystyle\frac}
\def \Inf{\displaystyle\inf}
\def \Sup{\displaystyle\sup}
\def \Lim{\displaystyle\lim}
\def \Liminf{\displaystyle\liminf}
\def \Limsup{\displaystyle\limsup}
\def \Max{\displaystyle\max}
\def \Min{\displaystyle\min}




\def \ni{\noindent}

\def \eps{\varepsilon}


\def \ep{\hbox{ }\hfill$\Box$}


\def\Dt#1{\Frac{\partial #1}{\partial t}}
\def\Dx#1{\Frac{\partial #1}{\partial x}}
\def\Ds#1{\Frac{\partial #1}{\partial s}}
\def\Dss#1{\Frac{\partial^2 #1}{\partial s^2}}
\def\Dy#1{\Frac{\partial #1}{\partial y}}
\def\Dyy#1{\Frac{\partial^2 #1}{\partial y^2}}
\def\Dsy#1{\Frac{\partial^2 #1}{\partial s \partial y}}
\def\Dk#1{\Frac{\partial #1}{\partial k}}
\def\Dp#1{\Frac{\partial #1}{\partial p}}
\def\Dkk#1{\Frac{\partial^2 #1}{\partial k^2}}
\def\Dpp#1{\Frac{\partial^2 #1}{\partial p^2}}
\def\Dky#1{\Frac{\partial^2 #1}{\partial k \partial y}}
\def\Dkp#1{\Frac{\partial^2 #1}{\partial k \partial p}}
\def\Dyp#1{\Frac{\partial^2 #1}{\partial y \partial p}}

\def\Dth#1{\Frac{\partial #1}{\partial \theta}}
\def\Dthi#1{\Frac{\partial #1}{\partial \theta_i}}
\def\Dthj#1{\Frac{\partial #1}{\partial \theta_j}}
\def\Dtth#1{\Frac{\partial^2 #1}{\partial \theta^2}}
\def\Dthij#1{\Frac{\partial^2 #1}{\partial \theta_i \partial \theta_j}}

\def\Dth#1{\Frac{\partial #1}{\partial \theta}}
\def\Dtth#1{\Frac{\partial^2 #1}{\partial \theta^2}}

\def\Dlam#1{\Frac{\partial #1}{\partial \lambda}}

\def\reff#1{{\rm(\ref{#1})}}

\def\beqs{\begin{eqnarray*}}
\def\enqs{\end{eqnarray*}}
\def\beq{\begin{eqnarray}}
\def\enq{\end{eqnarray}}


\usepackage[french]{babel}%typo française

\usepackage{times}




%%%%%\setbeamercovered{dynamic}






\newcounter{exo}
\def\cit{\addtocounter{exo}{-1}\refstepcounter{exo}\label}
\def\exo{\mbox{}\\[0em]\hspace*{0em}\bf Exercice
\addtocounter{exo}{1}\arabic{exo}.\rm\hspace{1ex}}

\parindent 0pt

%

\begin{document}
\centerline{\sc MSA}  \centerline{~}
\vskip1cm \centerline{{\bf TD 7}} \centerline{2018}

\exo On souhaite vérifier la qualité du générateur de nombres
aléatoires d'une calculatrice scientifique. Pour cela, on procède
à $250$ tirages dans l'ensemble $\{0,...,9\}$ et on obtient les
résultats suivants :

\beqs\begin{array}{ccccccccccc}
x & 0 &1&2&3&4&5&6&7&8&9 \\
N(x) & 28&32&23 &26 &23 &31 &18 &19 &19 &31
\end{array}
\enqs

A l'aide du test du $\chi^2$, vérifier si le générateur produit
des entiers indépendants et uniformément répartis sur
$\{0,...,9\}$.


\exo Pour déterminer si les merles vivent en communauté ou en
solitaire, on procède à l'expérience suivante: on dispose un filet
dans la zone d'habitat des merles, et on vient relever le nombre
de captures pendant 89 jours. On obtient les résultats suivants:

\beqs\begin{array}{cccccccc}
\hbox{Nombre de captures} & 0 &1&2&3&4&5&6 \\
\hbox{Nombre de jours} &56&22&9&1&0&1&0
\end{array}
\enqs

\ni {\bf 1.} On suppose qu'une loi de Poisson est représentative
de l'expérience. Construire un estimateur du paramètre de cette
loi.

\vspace{1mm}

\ni {\bf 2.} Vérifier à l'aide d'un test du $\chi^2$ l'adéquation
du modèle aux données. Faire l'application numérique au niveau
$\alpha = 5\%$.

\vspace{1mm}


\ni {\bf 3.} Reprendre l'exercice en groupant les catégories
Nombre de captures $= 2, 3, 4, 5$ et $6$ en Nombre de captures
$\geq 2$.

%\exo (Vecteur gaussien) Soit $X = (X_1,X_2,X_3,X_4)$ un vecteur
%gaussien centré de matrice de covariance
%
%
%\beqs\begin{array}{cccc}
%2& 1 & 0 &1\\
%1& 1 & 0& 1\\
%0& 0 & 1& 0\\
%1& 1 & 0& 2 \end{array} \enqs
%
%\ni {\bf 1.} Que peut-on dire de $X_3$ et de $(X_1,X_2,X_4)$?
%
%\vspace{1mm}
%
%\ni {\bf 2.} Donner la loi marginale de $(X_1,X_2)$ et calculer
%$E[X_1|X_2]$.
%
%\vspace{1mm}
%
%\ni {\bf 3.} Même question pour $(X_2,X_4)$.
%
%\vspace{1mm}
%
%\ni {\bf 4.} En déduire deux variables indépendantes de $X_2$,
%fonctions respectivement de $X_1, X_2$ et de $X_2,X_4$.
%
%\vspace{1mm}
%
%\ni {\bf 5.} Trouver une décomposition de $X$ en quatre vecteurs
%indépendants.

\end{document}

%\exo  (Vecteur gaussien) Soient $X$ et $Z$ deux variables
%aléatoires réelles indépendantes, $X$ étant de loi $\Nc(0, 1)$ et
%$Z$ de loi définie par $P(Z = 1) = P(Z = -1) = 1/2$. On pose $Y =
%ZX$.
%
%\ni {\bf 1.} Déterminer la loi de Y.
%
%\ni {\bf 2.} Calculer $Cov(X, Y )$.
%
%\ni {\bf 3.} Le vecteur $(X, Y)$ est-il gaussien? Les variables
%$X$ et $Y$ sont-elles indépendantes ?


\end{document}

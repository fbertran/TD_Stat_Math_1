
\documentclass[11pt,a4paper]{article}
\usepackage{a4wide}\usepackage{amsmath,amssymb}
\usepackage{dsfont}
\usepackage[utf8]{inputenc} % entree 8 bits iso-latin1
\usepackage[T1]{fontenc}      % encodage 8 bits des fontes utilisees
\usepackage[french]{babel}%typo française
\usepackage{times}
\usepackage{slashbox}
\newcommand{\R}{\mathbb{R}}\newcommand{\C}{\mathbb{C}}
\newcommand{\N}{\mathbb{N}}\newcommand{\Q}{\mathbb{Q}}
\usepackage{color}

\def \I{\mathbb{I}}
\def \N{\mathbb{N}}
\def \R{\mathbb{R}}
\def \M{\mathbb{M}}
\def \Z{\mathbb{Z}}
\def \E{\mathbb{E}}
\def \F{\mathbb{F}}
\def \P{\mathbb{P}}
\def \Q{\mathbb{Q}}
\def \D{\mathbb{D}}


\def \Ac{{\cal A}}
\def \Bc{{\cal B}}
\def \Cc{{\cal C}}
\def \Dc{{\cal D}}
\def \Ec{{\cal E}}
\def \Fc{{\cal F}}
\def \Gc{{\cal G}}
\def \Hc{{\cal H}}
\def \Ic{{\cal I}}
\def \Kc{{\cal K}}
\def \Lc{{\cal L}}
\def \Pc{{\cal P}}
\def \Rc{{\cal R}}
\def \Qc{{\cal Q}}
\def \Mc{{\cal M}}
\def \Nc{{\cal N}}
\def \Oc{{\cal O}}
\def \Sc{{\cal S}}
\def \Tc{{\cal T}}
\def \Uc{{\cal U}}
\def \Vc{{\cal V}}
\def \Wc{{\cal W}}
\def \Yc{{\cal Y}}
\def \Zc{{\cal Z}}
\def \Xc{{\cal X}}






\def \PI{\displaystyle\Pi}

\def \Sum{\displaystyle\sum}
\def \Prod{\displaystyle\prod}
\def \Int{\displaystyle\int}
\def \Frac{\displaystyle\frac}
\def \Inf{\displaystyle\inf}
\def \Sup{\displaystyle\sup}
\def \Lim{\displaystyle\lim}
\def \Liminf{\displaystyle\liminf}
\def \Limsup{\displaystyle\limsup}
\def \Max{\displaystyle\max}
\def \Min{\displaystyle\min}




\def \ni{\noindent}

\def \eps{\varepsilon}


\def \ep{\hbox{ }\hfill$\Box$}


\def\Dt#1{\Frac{\partial #1}{\partial t}}
\def\Dx#1{\Frac{\partial #1}{\partial x}}
\def\Ds#1{\Frac{\partial #1}{\partial s}}
\def\Dss#1{\Frac{\partial^2 #1}{\partial s^2}}
\def\Dy#1{\Frac{\partial #1}{\partial y}}
\def\Dyy#1{\Frac{\partial^2 #1}{\partial y^2}}
\def\Dsy#1{\Frac{\partial^2 #1}{\partial s \partial y}}
\def\Dk#1{\Frac{\partial #1}{\partial k}}
\def\Dp#1{\Frac{\partial #1}{\partial p}}
\def\Dkk#1{\Frac{\partial^2 #1}{\partial k^2}}
\def\Dpp#1{\Frac{\partial^2 #1}{\partial p^2}}
\def\Dky#1{\Frac{\partial^2 #1}{\partial k \partial y}}
\def\Dkp#1{\Frac{\partial^2 #1}{\partial k \partial p}}
\def\Dyp#1{\Frac{\partial^2 #1}{\partial y \partial p}}

\def\Dth#1{\Frac{\partial #1}{\partial \theta}}
\def\Dthi#1{\Frac{\partial #1}{\partial \theta_i}}
\def\Dthj#1{\Frac{\partial #1}{\partial \theta_j}}
\def\Dtth#1{\Frac{\partial^2 #1}{\partial \theta^2}}
\def\Dthij#1{\Frac{\partial^2 #1}{\partial \theta_i \partial \theta_j}}

\def\Dth#1{\Frac{\partial #1}{\partial \theta}}
\def\Dtth#1{\Frac{\partial^2 #1}{\partial \theta^2}}

\def\Dlam#1{\Frac{\partial #1}{\partial \lambda}}

\def\reff#1{{\rm(\ref{#1})}}

\def\beqs{\begin{eqnarray*}}
\def\enqs{\end{eqnarray*}}
\def\beq{\begin{eqnarray}}
\def\enq{\end{eqnarray}}


\usepackage[french]{babel}%typo française

\usepackage{times}




%%%%%\setbeamercovered{dynamic}






\newcounter{exo}
\def\cit{\addtocounter{exo}{-1}\refstepcounter{exo}\label}
\def\exo{\mbox{}\\[0em]\hspace*{0em}\bf Exercice
\addtocounter{exo}{1}\arabic{exo}.\rm\hspace{1ex}}

\parindent 0pt

%

\begin{document}
\centerline{\sc Modélisation statistique appliquée}  \centerline{~}
\vskip1cm \centerline{{\bf TD 8}} \centerline{2018}

\exo Test de la liaison de deux caractères : couleur des yeux et des cheveux\\
Le tableau de contingence suivant indique le résultat de l'examen de 124 sujets, classés d'après la couleur de leurs yeux et la couleur de leurs cheveux.

\begin{center}
\begin{tabular}{||c||c|c|c|c||}
\hline
\hline
\backslashbox{Couleur yeux}{Couleur cheveux}& blonds& bruns& noirs& roux\\
\hline
\hline
bleus&25&9&3&7\\
\hline
gris ou verts&13&17&10&7\\
\hline
marrons&7&13&8&5\\
\hline
\hline
\end{tabular}
\end{center}
Tester s'il existe une liaison entre ces deux caractères.

\exo Générateur aléatoire\\
Décider, au seuil de signification de 5\% et avec le test le Kolmogorov, si les nombres suivants sont susceptibles d'être des réalisations d'une loi uniforme sur l'intervalle $[0;1]$.
\begin{center}
\begin{tabular}{||c|c|c|c|c|c||}
\hline
\hline
0,07117832&0,03982727&0,13429693&0,61832706&0,37390333&0,37641346\\
\hline
0,73474129&0,89435507&0,65742160&0,03694860&0,25869665&0,46339134\\
\hline
0,02668867&0,34375192&0,03395492&0,43945444&0,52811985&0,88284524\\
\hline
0,73236028&0,76711914&&&&\\
\hline
\hline
\end{tabular}
\end{center}

\end{document}

%\exo  (Vecteur gaussien) Soient $X$ et $Z$ deux variables
%aléatoires réelles indépendantes, $X$ étant de loi $\Nc(0, 1)$ et
%$Z$ de loi définie par $P(Z = 1) = P(Z = -1) = 1/2$. On pose $Y =
%ZX$.
%
%\ni {\bf 1.} Déterminer la loi de Y.
%
%\ni {\bf 2.} Calculer $Cov(X, Y )$.
%
%\ni {\bf 3.} Le vecteur $(X, Y)$ est-il gaussien? Les variables
%$X$ et $Y$ sont-elles indépendantes ?


\end{document}
